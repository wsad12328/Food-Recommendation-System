\section{Conclusion}
我們提出了一種新穎的方法,利用NIE-GCN模型來解決餐廳推薦系統的挑戰。該模型能有效從二分異質圖中提取更深層的關聯信息,並精確預測顧客節點最感興趣的前 $k$ 個餐廳節點。我們的方法在推薦系統中表現出色,Recall@5 達到 0.810,NDCG@5 則達到 0.917,顯示出該模型在推薦任務中的優越性能。

本研究的主要挑戰在於資料集的建立。由於無法透過爬蟲從 Foodpanda 和 Google Maps 等平台獲取用戶與餐廳的互動資料,我們採用了問卷調查的方式收集資料。但此方法導致資料量不足,且無法完全保證資料的真實性。問卷調查無法直接獲得用戶與餐廳的互動資料,因此我們只能藉由用戶填寫的興趣類別來模擬實際的互動關係。此外,如何將每個店家的評論與評分轉換為適合分析的特徵向量,也是我們面臨的一大挑戰。研究結果顯示,NIE-GCN模型在提升餐廳推薦系統性能方面具有巨大潛力。未來的研究應著重於建立更大規模且更具真實性的資料集,例如與業界合作,進一步驗證並優化所提出的方法。
\color{black}